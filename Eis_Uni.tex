\documentclass[12pt]{article}
%\usepackage{etex} 
\usepackage[T1]{fontenc}
%\usepackage{palatino}
\usepackage{amsmath,amsthm,amssymb}
%\usepackage{mathpple}
%\usepackage{upgreek}
\usepackage{mathpazo}
\usepackage{accents}          
%\usepackage{newpxtext,newpxmath}
\usepackage{indentfirst}
\setlength{\parindent}{1.5em}
\usepackage{mathrsfs}
\usepackage{graphicx}
\usepackage{tikz}
\usepackage{tikz-cd}
\usepackage{tabularx} % extra features for tabular environment
\usepackage[margin=1in]{geometry} % decreases margins
\usepackage{dynkin-diagrams}
\usepackage{blindtext}
\usepackage{mathtools}
\usepackage{graphicx} % support the \includegraphics command and options
%\usepackage{hyperref}
\usepackage[unicode=true,bookmarks=true,bookmarksnumbered=false,bookmarksopen=false, breaklinks=false,pdfborder={0 0 1},backref=false,colorlinks=true,citecolor=violet!50!white,linkcolor=green]{hyperref}
\usepackage{csquotes}
%\usepackage[parfill]{parskip} % Activate to begin paragraphs with an empty line rather than an indent
\usepackage{booktabs} % for much better looking tables
\usepackage{array} % for better arrays (eg matrices) in maths
\usepackage{paralist} % very flexible & customisable lists (eg. enumerate/itemize, etc.)
\usepackage{subfig} % make it possible to include more than one captioned figure/table in a single float
\usepackage{amsfonts, bbm, dsfont}
\usepackage{stmaryrd}
\usepackage{float}
\usepackage{enumitem}
\usepackage{url}
\usepackage{tkz-graph}
%\usepackage[mathcal]{euler}
%\usepackage{eucal}  
\usepackage{multirow}
\usepackage{titlesec}
\usepackage{cleveref}
%\titleformat{\section}[runin]{\normalfont\Large\bfseries}{\thesection}{1em}{}
%\titleformat{\subsection}[runin]{\normalfont\large\bfseries}{\thesubsection}{1em}{}
\usepackage{accents}
\usepackage[pdf]{pstricks} 
\usepackage[most]{tcolorbox}

\crefformat{section}{\S#2#1#3} % see manual of cleveref, section 8.2.1
\crefformat{subsection}{\S#2#1#3}
\crefformat{subsubsection}{\S#2#1#3}

\setlist[enumerate]{label=(\arabic*)}  % all levels
\setlist[enumerate,2]{label=(\roman*)}
\setlist[enumerate,3]{label=(\alph*)}
\renewcommand\labelitemi{---}

%\renewcommand*{\EdgeLineWidth}{0.15pt}

\usepackage[backend=biber,style=alphabetic,maxnames=99,maxalphanames=10]{biblatex}
\AtEveryBibitem{
    \clearfield{urlyear}
    \clearfield{urlmonth}
}
\bibliography{bibliography.bib}
%\addbibresource{bibliography.bib}

\NewBibliographyString{diplomathesis}
\DefineBibliographyStrings{english}{diplomathesis = {diploma thesis},}

\pgfkeys{/Dynkin diagram,
edge length=0.8cm,
fold radius=0.8cm,
root radius=0.1cm,
indefinite edge/.style={
draw=black,
fill=white,
thin,
densely dashed}}

% --------------------------------------------------------------
%                         Start here
% --------------------------------------------------------------
\newtheorem{thm}{Theorem}[section]
\newtheorem{cor}[thm]{Corollary}
\newtheorem{lemma}[thm]{Lemma}
\newtheorem{prop}[thm]{Proposition}
\newtheorem{conj}[thm]{Conjecture}
\newtheorem{prob}[thm]{Problem}
\theoremstyle{remark}
\newtheorem{ex}[thm]{Example}
\newtheorem{const}[thm]{Construction}
\newtheorem{exer}[thm]{Exercise}
\newtheorem{rmk}[thm]{Remark}
\newtheorem{ques}[thm]{Question}
\newtheorem*{warn}{\textcolor{red}{Warning}}
\theoremstyle{definition}
\newtheorem{defi}[thm]{Definition}


\newcommand{\dd}{\mathrm{d}}
\newcommand{\norm}{\mathrm{N}}
\newcommand{\sym}{\operatorname{Sym}}
\newcommand{\spin}{\operatorname{Spin}}
\newcommand{\lra}{\longrightarrow}
\newcommand{\alg}{\mathrm{alg}}
\newcommand{\catset}{\mathsf{Set}}
\newcommand{\spec}{\operatorname{Spec}}
\newcommand{\Hom}{\mathrm{Hom}}
\newcommand{\mor}{\mathbf{Mor}}
\newcommand{\oo}{\mathcal{O}}
\newcommand{\sch}{\mathsf{Sch}}
\newcommand{\affsch}{\mathsf{AffSch}}
\newcommand{\falg}{\mathsf{fAlg}}
\newcommand{\grp}{\mathsf{Grp}}
\newcommand{\Z}{\mathbb{Z}}
\newcommand{\C}{\mathbb{C}}
\newcommand{\gal}{\mathrm{Gal}}
\newcommand{\Q}{\mathbb{Q}}
\newcommand{\F}{\mathbb{F}}
\newcommand{\oct}{\mathbb{O}}
\newcommand{\coct}{\mathscr{R}}
\newcommand{\R}{\mathbb{R}}
\newcommand{\A}{\mathbb{A}}
\newcommand{\tr}{\operatorname{Tr}}
\newcommand{\frob}{\mathrm{Frob}}
\newcommand{\aut}{\mathrm{Aut}}
\newcommand{\st}{\mathrm{St}}
\newcommand{\modulo}{\operatorname{mod}}
\newcommand{\cusp}{\mathrm{cusp}}
\newcommand{\disc}{\mathrm{disc}}
\newcommand{\weights}{\mathrm{Weights}}
\newcommand{\Gtwo}{\mathrm{G}_{2}}
\newcommand{\lie}{\operatorname{Lie}}
\newcommand{\Ind}[2]{\operatorname{Ind}_{#1}^{#2}}
\newcommand{\ind}[2]{\operatorname{ind}_{#1}^{#2}}
\newcommand{\hind}[2]{\mbox{$\mathrm{h}$-$\operatorname{ind}_{#1}^{#2}$}}
\newcommand{\lietype}[2]{\operatorname{#1}_{#2}}
\newcommand{\smallmat}[4]{\left(\begin{smallmatrix}#1&#2\\#3&#4\end{smallmatrix}\right)}
\newcommand{\mat}[4]{\left(\begin{matrix}#1&#2\\#3&#4\end{matrix}\right)}
\newcommand{\lrangle}[2]{\langle #1,#2\rangle}
\newcommand{\bracket}[1]{\left(#1\right)}
\newcommand{\spa}[2]{\operatorname{Spa}\left(#1,#2\right)}
\newcommand{\et}{\acute{e}t}
\newcommand{\grade}[1]{\operatorname{gr}#1}
\newcommand{\pfrac}[2]{\frac{\partial #1}{\partial #2}}
\newcommand{\overcirc}[1]{\accentset{\circ}{#1}}
\newcommand{\Nu}{\mathcal{V}}
\newcommand{\isom}{\overset{\sim}{\rightarrow}}
\newcommand{\rmm}[1]{\mathrm{#1}}
\newcommand{\bff}[1]{\mathbf{#1}}
\newcommand{\triv}{\mathbb{1}}
\newcommand{\midline}{\,\middle\vert\,}
\newcommand{\set}[2]{\left\{#1\midline #2\right\}}
\newcommand{\vrep}[1]{\mathrm{V}_{#1}}
\newcommand{\cstar}{$C^{*}$}
\newcommand{\funspace}[2]{\call{C}_{#1}^{#2}}
\newcommand{\ad}{\operatorname{ad}}
\newcommand{\Ad}{\operatorname{Ad}}
\newcommand{\call}[1]{\mathcal{#1}}
\newcommand{\frakk}[1]{\mathfrak{#1}}
\newcommand{\ltwo}[1]{\rmm{L}^{2}(#1)}
\newcommand{\ifff}{if and only if}
\newcommand{\mbb}[1]{\mathbb{#1}}
\newcommand{\redcstar}[1]{\funspace{r}{*}(#1)}
\newcommand{\avgfin}[2]{\frac{1}{|#2|}\sum_{#1 \in #2}}
\newcommand{\rank}{\operatorname{rank}}
\newcommand{\refbox}[1]{\begin{tcolorbox}[colback=violet!5!white,colframe=violet!50!white,title=References]#1\end{tcolorbox}}

\newcommand{\GL}{\mathbf{GL}}
\newcommand{\LieGL}{\mathrm{GL}}
\newcommand{\SL}{\mathbf{SL}}
\newcommand{\LieSL}{\mathrm{SL}}
\newcommand{\Orth}{\mathbf{O}}
\newcommand{\LieOrth}{\operatorname{O}}
\newcommand{\Sorth}{\mathbf{SO}}
\newcommand{\LieSorth}{\operatorname{SO}}
\newcommand{\Symp}{\mathbf{Sp}}
\newcommand{\LieSymp}{\operatorname{Sp}}
\newcommand{\Pgl}{\mathbf{PGL}}
\newcommand{\LiePgl}{\operatorname{PGL}}
\newcommand{\grpF}{\bff{F}_{4}}
\let\oldemptyset\emptyset
\let\emptyset\varnothing

%\DeclarePairedDelimiter\ceil{\lceil}{\rceil}
%\DeclarePairedDelimiter\floor{\lfloor}{\rfloor}
%++++++++++++++++++++++++++++++++++++++++
\setcounter{tocdepth}{4}
\setcounter{secnumdepth}{4}

\begin{document}
\title{Quaternionic Degenerate Eisenstein series on \texorpdfstring{$\bff{U}(2,n)$}{}}
\author{}
\date{\today}
\maketitle
\begin{abstract}
    tba
\end{abstract}
\section{Unitary groups of split rank 2}
\label{section rank 2 unitary groups}
\textcolor{red}{
If some notation appearing in this section is not defined,
then it should be defined in \cite{Hilado_McGlade_Yan_Fourier_coefficients_unitary_group}.
}

Let $E$ be an imaginary quadratic extension of $\Q$,
and we denote the nonzero element in $\mathrm{Gal}(E/\Q)$ by $c$ or $x\mapsto \overline{x}$.
Write the norm of $E/\Q$ as $|\,|$.

Define $\bff{G}=\bff{U}(V)=\bff{U}(2,n)$ to be the unitary group of a non-degenerate Hermitian space $(V,\lrangle{\,}{\,})$ over $E$ with signature $(2,n)$.
We write the $\bff{G}$-action on $V$ as a right action.

Fix a pair of isotropic lines $(U,U^{\vee})$ inside $V$ such that $\lrangle{U}{U^{\vee}}\neq 0$,
and take $V_{0}$ to be the orthogonal complement of $U\oplus U^{\vee}$,
which is a Hermitian space of signature $(1,n-1)$.
Fix $b_{1}\in U$ and $b_{2}\in U^{\vee}$ such that $\lrangle{b_{1}}{b_{2}}=1$.

One can define a parabolic subgroup $\bff{P}$ of $\bff{G}$
as the stabilizer of $U$, 
and it has the following realization of Levi decomposition $\bff{P}=\bff{MN}$:
\begin{itemize}
    \item $\bff{M}$ is the stabilizer of $U$ and $U^{\vee}$, and it is isomorphic to $\rmm{Res}_{E/\Q}(\bff{G}_{\rmm{m}})\times\bff{U}(V_{0})$.
    One can write the $\bff{M}$-action on $(u,w,u^{\vee})\in U\oplus V_{0}\oplus U^{\vee}$ explicitly:
    \[(u,w,u^{\vee}).(z,h)=(z^{-1}u,wh,\overline{z}u^{\vee}).\]
    \item $\bff{N}$ is a Heisenberg, and it is isomorphic to the group 
    \[\set{(v,\lambda)\in V_{0}\times \rmm{Res}_{E/\Q}\bff{G}_{\rmm{a}}}{\overline{\lambda}=-\lambda},\]
    equipped with the following multiplication:
    \[(v_{1},\lambda_{1})\cdot(v_{2},\lambda_{2}):=\left(v_{1}+v_{2},\lambda_{1}+\lambda_{2}-\frac{\lrangle{v_{1}}{v_{2}}-\lrangle{v_{2}}{v_{1}}}{2}\right).\]
    The action of the corresponding element $n(v,\lambda)\in\bff{N}$ on $V$ is given as: 
    \[b_{1}\mapsto b_{1},\,b_{2}\mapsto (-\frac{1}{2}\lrangle{v}{v}+\lambda)b_{1}+b_{2}+v,\,w\in V_{0}\mapsto -\lrangle{w}{v}b_{1}+w.\]
    One has 
    \[(z,h)n(v,\lambda)(z^{-1},h^{-1})=n(\overline{z}vh^{-1},|z|\lambda).\]
\end{itemize}
Denote the center of $\bff{N}$ by $\bff{N}_{0}$.
We denote by $\nu$
the similitude character of $\bff{M}$: $\nu(z,h)=z\in \rmm{Res}_{E/\Q}\bff{G}_{\rmm{m}}$,
so that $b_{1}(z,h)=\nu(z,h)^{-1}b_{1}$.

Fix an addictive character $\psi:\Q\backslash\A\rightarrow \C^{\times}$,
so that $\psi_{p}$ has conductor $\Z_{p}$
and $\psi_{\infty}(x)=e^{2\pi i x}$.
We have the following identification $V_{0}\xrightarrow{\sim}\Hom([\bff{N}],\mbb{S}^{1})$:
for any element $T\in V_{0}$,
one associates a unitary character $\chi_{T}$ of $[\bff{N}]$ by 
\[\chi_{T}:[\bff{N}]\rightarrow \C^{\times},\,n(v,\lambda)\mapsto \psi(-\rmm{Im}\lrangle{T}{v}).\]
\textcolor{red}{I am really confused with this identification in \cite{Hilado_McGlade_Yan_Fourier_coefficients_unitary_group}. 
This character $\chi_{T}$ seems not well-defined:
the symplectic form $-\rmm{Im}\lrangle{T}{v}=\frac{\lrangle{v}{T}-\lrangle{T}{v}}{2i}$ involves $1/2i$, which may not lie in $E$.
Maybe one should define $\chi_{T}$ to be $n(v,\lambda)\mapsto \psi(\rmm{Re}\lrangle{T}{v})$.}

For a finite place $p$ of $\Q$,
we define $E_{p}$ to be $E_{p}:=\Q_{p}\otimes_{\Q}E$,
which is isomorphic to
\begin{itemize}
    \item $\Q_{p}\times \Q_{p}$, if $p$ splits in $E$;
    \item a degree $2$ unramified (\emph{resp.\,}ramified) extension of $\Q_{p}$, if $p$ is inert (\emph{resp.\,}ramified).
\end{itemize}


\section{Heisenberg Eisenstein series}
Choose a section $f=f_{\ell,\infty}\otimes f_{fte}$ of $\rmm{Ind}_{\bff{P}(\A)}^{\bff{G}(\A)}|\nu|^{s}$ (unnormalized) as follow:
\begin{itemize}
    \item $f_{\ell,\infty}$ is the $\mbb{V}_{\ell}=\left(\sym^{2\ell}V_{2}^{+}\otimes \det^{-\ell}_{\bff{U}(2)}\right)\boxtimes\bff{1}$-valued, $K_{\infty}$-equivariant induced section
    whose restriction to $\bff{M}(\R)$ is 
    \[f_{\ell,\infty}((z,h),s)=|z|^{s}[u_{1}^{\ell}][u_{2}^{\ell}],\]
    where $[u_{i}^{k}]:=u_{i}^{k}/k!$.
    \item $f_{fte}$ is defined as 
    \[f_{fte}(g_{f},\Phi_{f},s)=\int_{\GL_{1}(\A_{E,f})}|t|^{s}\Phi_{f}(t\cdot b_{1}g_{f})dt,\]
    where $\Phi_{f}=\prod \Phi_{p}$ is a Schwartz-Bruhat function on $V(\A_{f})$.
    %Here we assume that $\Phi_{p}$ is the characteristic function of the lattice $\oo_{E,p}b_{1}\oplus V_{0}(\oo_{E,p})\oplus\oo_{E,p}b_{2}$,
    %where $\oo_{E,p}=\Z_{p}\otimes_{\Z}\oo_{E}$ and $V_{0}(\oo_{E,p})$ is some lattice of $V_{0}\otimes_{\Q_{p}}\Q$ (\textcolor{red}{precise definition to be added}). 
\end{itemize}
One defines the degenerate Heisenberg Eisenstein series: 
\[E_{\ell}(g,\Phi_{f},s)=\sum_{\gamma\in \bff{P}(\Q)\backslash \bff{G}(\Q)}f(\gamma g,s).\]
To prove the modularity of this Eisenstein series,
one needs the following result:
\begin{prop}\label{prop archimedean section killed by Schmid operators}
    The $\mbb{V}_{\ell}$-valued section $f_{\ell,\infty}$ of $\rmm{Ind}_{\bff{P}(\R)}^{\bff{G}(\R)}|\nu|^{s}$,
    as a function on $\bff{G}(\R)$,
    is killed by the Schmid operators $\call{D}_{\ell}^{\pm}$
    if and only if $s=\ell+1$.
\end{prop}
\begin{proof}
    Write an element $(z,h)\in\bff{M}(\R)$ as $m=(h,r,\theta)$ so that $z=re^{i\theta},\,r\in\R_{>0}$ and $\theta\in [0,2\pi)$.
    Using this coordinates,
    we define a function $F$ on $\bff{M}(\R)$,
    sending $(h,r,\theta)$ to $r^{s}$.
    By \cite[Proposition 3.10]{Hilado_McGlade_Yan_Fourier_coefficients_unitary_group},
    $f_{\ell,\infty}$ is killed by $\call{D}_{\ell}^{\pm}$ if and only if 
    \[\left\{\begin{array}{l}
            (r\partial r-2(\ell+1))F(m)=0\\[6pt]
            [u_{2}\otimes\overline{v_{k}}]^{+}F(m)=0,\text{ if } 1 \leq k<n\\[6pt]
            [u_{2}\otimes\overline{v_{k}}]^{-}F(m)=0,\text{ if }1\leq k<n
        \end{array}\right.\]
    It can be easily verified that these equations hold if and only if $s=\ell+1$.
\end{proof}
As a consequence,
the Eisenstein series $E_{\ell}(g,\Phi_{f},s=\ell+1)$ is a quaternionic modular form of weight $\ell$.
\section{The Fourier expansion of \texorpdfstring{$E_{\ell}(g,\Phi_{f},s=\ell+1)$}{PDFstring}}
\label{section Fourier expansion of Eisenstein series}
\subsection{Abstract Fourier expansion}
\label{section abstract Fourier expansion}
In this subsection,
we give the ``abstract'' Fourier expansion of $E_{\ell}(g,\Phi_{f},s)$.
\begin{lemma}\label{lemma double coset of parabolic of unitary groups}
    The right $\bff{P}(\Q)$-space $\bff{P}(\Q)\backslash \bff{G}(\Q)$,
    the space of isotropic lines in $V(\Q)$,
    has exactly $3$ orbits of $\bff{P}(\Q)$,
    represented respectively by $\Q b_{1}$, $\Q v_{0}$ and $\Q b_{2}$,
    where $v_{0}$ is an arbitrary non-zero isotropic vector in $V_{0}(\Q)$.
\end{lemma}
\begin{proof}
    Directly by the explicit action given in \Cref{section rank 2 unitary groups}.
\end{proof}
Set $\bff{G}(\Q)=\bigsqcup_{i=0}^{2}\bff{P}(\Q)w_{i} \bff{P}(\Q)$,
such that $w_{0}=1$,
$b_{1}w_{1}=v_{0}$ and $b_{1}w_{2}=b_{2}$.
Now we can write the degenerate Eisenstein series as 
\[E_{\ell}(g,\Phi_{f},s)=\sum_{i=0}^{2}E_{\ell,i}(g,\Phi_{f},s),\,E_{\ell,i}(g,\Phi_{f},s)=\sum_{\gamma\in \bff{P}(\Q)\backslash \bff{P}(\Q)w_{i}\bff{P}(\Q)}f(\gamma g,s),\]
thus $E_{\ell,0}(g,\Phi_{f},s)=f(g,s)$.
From now on,
when there is no confusion
we will omit the $\ell$ and $\Phi_{f}$ in $E_{\ell}(g,\Phi_{f},s)$,
and write it as $E(g,s)=\sum_{i=0}^{2}E_{i}(g,s)$.

\begin{lemma}
    \label{lemma rank decomposition of Eisenstein series}
    Assume that $\rmm{Re}(s)\gg 0$ so that the sum defining $E(g,s)$ converges absolutely.
    Then one has the following expressions for the $E_{i}(g,s)$:
    \begin{enumerate}
        \item Let $\call{L}_{0}$ be the set of non-zero isotropic lines $\ell$ in $V_{0}$ 
        and for any $\ell\in\call{L}_{0}$, select $\gamma(\ell)\in \bff{G}(\Q)$ with $b_{1}\gamma(\ell)\in\ell$.
        Then \[E_{1}(g,s)=\sum_{\ell\in\call{L}_{0}}\sum_{\mu\in(\ell)^{\perp}\bff{N}_{0}(\Q)\backslash\bff{N}(\Q)}f(\gamma(\ell)\mu g,s).\] 
        \item One has 
        \[E_{2}(g,s)=\sum_{\mu\in \bff{N}(\Q)}f(w_{2}\mu g,s).\]
    \end{enumerate}
\end{lemma} 

For any $T\in V_{0}$,
we set 
\[E_{i}^{T}(g,s)=\int_{\bff{N}(F)\backslash\bff{N}(\A)}\chi_{T}^{-1}(n)E_{i}(ng,s)dn,\,i=0,1,2.\]

\begin{lemma}
    \label{lemma Eulerian Fourier coefficients}
    \begin{enumerate}
        \item If $T$ is anisotropic, then $E_{1}^{T}=0$. 
            If $T$ is isotropic, define $\bff{N}_{T}=(\ell_{T})^{\perp}\bff{N}_{0}\subseteq \bff{N}$,
            then 
            \[E_{1}^{T}(g,s)=\int_{\bff{N}_{T}(\A)\backslash \bff{N}(\A)}\chi_{T}^{-1}(n)f(\gamma(\ell_{T})ng,s)dn.\]
        \item For any $T\in V_{0}$, one has 
        \[E_{2}^{T}(g,s)=\int_{\bff{N}(\A)}\chi_{T}^{-1}(n)f(w_{2}ng,s)dn.\]
    \end{enumerate}
\end{lemma}
\begin{proof}
    It suffices only to prove the $i=1$ case.
    For any $\ell\in \call{L}_{0}$,
    set \[\bff{N}_{\ell}=\set{n(v,\lambda)\in\bff{N}}{v\in\ell^{\perp}}\subseteq \bff{N}.\]
    For any $T\in V_{0}$,
    \begin{align*}
        E_{1}^{T}(g,s)=&\sum_{\ell\in\call{L}_{0}}\int_{[N]}\chi_{T}^{-1}(n)\left(\sum_{\mu\in\bff{N}_{\ell}(\Q)\backslash \bff{N}(\Q)}f(\gamma(\ell)\mu ng,s)\right)dn\\
        =&\sum_{\ell\in\call{L}_{0}}\int_{\bff{N}_{\ell}(\Q)\backslash\bff{N}(\A)}\chi_{T}^{-1}(n)f(\gamma(\ell)ng,s)dn\\
        =&\sum_{\ell\in\call{L}_{0}}\int_{\bff{N}_{\ell}(\A)\backslash\bff{N}(\A)}\left(\int_{[\bff{N}_{\ell}]}\chi_{T}^{-1}(r)dr\right)\chi_{T}^{-1}(n)f(\gamma(\ell)ng,s)dn\\
        =&\sum_{\ell\in\call{L}_{0},\,\chi_{T}|_{\bff{N}_{\ell}}\equiv 1}\int_{\bff{N}_{\ell}(\A)\backslash \bff{N}(\A)}\chi_{T}^{-1}(n)f(\gamma(\ell)ng,s)dn.
    \end{align*}
    Then the lemma follows from the fact that $\chi_{T}|_{\bff{N}_{\ell}}\equiv 1$ if and only if $T\in\ell$, \emph{i.e.\,}$T$ is isotropic and $\ell=\ell_{T}$. 
\end{proof}
%Before calculating the Fourier coefficients explicitly,
%we make the following assumptions on the Schwartz-Bruhat function $\Phi_{f}$:
%\begin{itemize}
%    \item $\Phi_{f}$ is a $\Q$-valued function.
%    \item There exists Schwartz-Bruhat functions $\Phi^{U},\Phi^{U^{\vee}}$ on $\A_{f}$ and $\Phi^{0}$ on $V_{0}(\A_{f})$ so that 
%    $\Phi_{f}(\alpha b_{1}+v_{0}+\beta b_{2})=\Phi^{U}(\alpha)\Phi^{0}(v_{0})\Phi^{U^{\vee}}(\beta)$, where $\alpha,\beta\in\A_{f}$ and $v_{0}\in V_{0}(\A_{f})$.
%    \item The functions $\Phi^{?}$ with $?\in\{U,U^{\vee},0\}$ satisfy $\Phi^{?}(\mu x)=\Phi^{?}(x)$ for all $\mu\in\widehat{\Z}^{\times}$.
%    \item $\Phi^{U}=\Phi^{U^{\vee}}$ is the characteristic function on $\widehat{\Z}$.
%    \item \textcolor{red}{Maybe more assumptions? To be added when needed.}
%\end{itemize}
\subsection{Computation of constant term}
\label{section constant term of Eisenstein series}
\textcolor{red}{I think we also need a refined version of \cite[Corollary 1.2]{Hilado_McGlade_Yan_Fourier_coefficients_unitary_group},
with a description on its constant term.}
\subsubsection{The \texorpdfstring{$i=0$}{PDFstring}-term}
\label{section i=0 part constant term}
\begin{lemma}
    For $g\in \bff{P}(\A)$,
    \[E_{0}(g,s)=f(g,s)=|\nu(g)|^{s}\zeta_{E}(s)[u_{1}^{n}][u_{2}^{n}].\]
\end{lemma}
\begin{proof}
    For $g_{f}\in \bff{P}(\A_{f})$,
    we have 
    \begin{align*}
        f_{fte}(g_f,s)&=\int_{\A_{E,f}^{\times}}|t|^{s}\Phi_{f}(tb_{1}g_{f})dt\\
        &=\int_{\A_{E,f}^{\times}}|t|^{s}\Phi_{f}(t\nu(g_{f})^{-1}b_{1})dt\\
        &=|\nu(g_{f})|^{s}\int_{\A_{E,f}^{\times}}|t|^{s}\Phi_{f}(tb_{1})dt.
    \end{align*}
    Thus, the non-archimedean contribution is $|\nu(g_{f})|^{s}\zeta_{E}(s)$.
    Combining with $f_{n,\infty}(g_{\infty},s)=|\nu(g_{\infty})|^{s}[u_{1}^{n}][u_{2}^{n}]$,
    we get the desired identity.
\end{proof}
\subsubsection{The \texorpdfstring{$i=1$}{PDFstring}-term}
\label{section i=1 part constant term}
We fix a non-zero isotropic vector in $V_{0}$,
such that $v_{0}=b_{1}\gamma_{0}$,
and set $\ell_{0}=Ev_{0}$.
Define $\bff{P}_{0}$ be the stabilizer of $\ell_{0}$ in $\bff{U}(V_{0})$,
which is a parabolic subgroup of $\bff{M}$.
We denote the similitude character of $\bff{P}_{0}$ by $\lambda$,
\emph{i.e.\,}$v_{0}g=\lambda(g)^{-1}v_{0}$ for any $g\in\bff{P}_{0}$.
For $g\in\bff{P}_{0}(\A)$,
we have:
\begin{align*}
    E_{1}^{0}(g,s)&=\sum_{\ell\in \call{L}_{0}}\int_{\bff{N}_{\ell}(\A)\backslash \bff{N}(\A)}f(\gamma(\ell)ng,s)dn \\
    &=\sum_{\gamma\in \bff{P}_{0}(\Q)\backslash\bff{M}(\Q)}\int_{\bff{N}_{\ell_{0}}(\A)\backslash \bff{N}(\A)}f(\gamma_{0}\gamma ng,s)dn
\end{align*}
If we set $f_{0}(g,s)=\int_{\bff{N}_{\ell_{0}}(\A)\backslash \bff{N}(\A)}f(\gamma_{0}ng,s)dn,$
then for $\rmm{Re}(s)\gg 0$,
\[E_{1}^{0}(g,s)=\sum_{\gamma\in\bff{P}_{0}(\Q)\backslash \bff{M}(\Q)}f_{0}(\gamma g,s),\]
and it defines an Eisenstein series on $\bff{M}$.
Now we want to determine this section $f_{0}(g,s)\in\rmm{Ind}_{\bff{P}_{0}(\A)}^{\bff{M}(\A)}|\lambda|^{s}$ (still unnormalized).
%\begin{prop}\label{prop i=1 contribution of constant term}
%    s
%\end{prop}
%\begin{proof}
   
At finite places, one has 
   \begin{align*}
       \int_{\bff{N}_{\ell_{0}}(\A_{E,f})\backslash\bff{N}(\A_{E,f})}f(\gamma_{0}ng,s)dn&=\int_{x\in\A_{E,f}}\int_{t\in\A_{E,f}^{\times}}|t|^{s}\Phi_{fte}(t(v_{0}+xb_{1})g)dtdx\\
       &=\int_{x\in\A_{E,f}}\int_{t\in\A_{E,f}^{\times}}|t|^{s}\Phi_{fte}(t\lambda(g)^{-1}v_{0}+tx\nu(g)^{-1}b_{1})dtdx\\
       &=|\lambda(g)|_{f}^{s-1}|\nu(g)|_{f}\int_{x\in\A_{E,f}}\int_{t\in\A_{f,E}^{\times}}|t|^{s-1}\Phi_{fte}(tv_{0}+xb_{1})dtdx\\
       &=|\lambda(g)|_{f}^{s-1}|\nu(g)|_{f}\zeta_{E}(s-1).
   \end{align*}
   Now we switch to the archimedean place.
   Set $c_{1}=v_{0}$ and $c_{2}$ another isotropic vector in $V_{0}$ with $\lrangle{c_{1}}{c_{2}}=1$.
   We can take $u_{2}=\frac{1}{\sqrt{2}}(c_{1}+c_{2})$ and $v_{n-1}=\frac{1}{\sqrt{2}}(c_{1}-c_{2})$.
   \begin{lemma}
    We have 
    \[f_{0,\infty}(g,s)=\]
   \end{lemma}
   \begin{proof}
    One picks the following representatives for $\bff{N}_{\ell_{0}}(\R)\backslash\bff{N}(\R)$: 
   $\left\{n(-xc_{2},0),\,x\in\R \right\}$,
   thus 
   $b_{1}\gamma_{0}n(-xc_{2},0)g=\lambda(g)^{-1}c_{1}+x\nu(g)^{-1}b_{1}$.
    Suppose that $\gamma_{0}n(-xc_{2},0)g=pk$ for some $p\in\bff{P}(\R)$ and $k=(k^{+},k^{-})\in K_{\infty}$,
    then one has 
    \[\lambda(p)^{-1}b_{1}k=\lambda(g)^{-1}c_{1}+x\nu(g)^{-1}b_{1}.\]
    Projecting both sides to $V_{2}^{+}$,
    one has 
    \[\lambda(p)^{-1}u_{1}k^{+}=x\nu(g)^{-1}u_{1}+\lambda(g)^{-1}u_{2}.\]
    So one can take $p$ such that 
    \[\lambda(p)^{-1}=\sqrt{|\lambda(g)|^{-2}+x^{2}|\nu(g)|^{-2}}\]
    and 
    \[k^{+}=\lambda(p)\left(\begin{matrix}
        x\nu(g)^{-1} & -\overline{\lambda(g)}^{-1}\\
        \lambda(g)^{-1} & x\overline{\nu(g)}^{-1}
    \end{matrix}\right)
    .
    \]
    Hence we have 
    \begin{align*}
        f_{\ell,\infty}(\gamma_{0}n(-xc_{2},0)g)&=f_{\ell,\infty}(pk)\\
        &=|\nu(p)|^{s}([u_{1}^{\ell}][u_{2}^{\ell}])k\\
        &=\frac{(x\nu(g)^{-1}u_{1}+\lambda(g)^{-1}u_{2})^{\ell}(-\overline{\lambda(g)}^{-1}u_{1}+x\overline{\nu(g)}^{-1}u_{2})^{\ell}}{(|\lambda(g)|^{-2}+x^{2}|\nu(g)|^{-2})^{s+\ell}(\ell!)^{2}}\\
    \end{align*}
    The coefficient of $[u_{1}^{\ell-v}][u_{2}^{\ell+v}]$ in $f_{\ell,\infty}(\gamma_{0}n(-xc_{2},0)g)$ is 
    \begin{align*}
        &\frac{(\ell-v)!(\ell+v)!}{(\ell!)^{2}(|\lambda|^{-2}+x^{2}|\nu|^{-2})^{2\ell+1}}\sum_{\substack{0\leq i,j\leq \ell\\ i+j=\ell-v}}\binom{\ell}{i}\binom{\ell}{j}x^{i}\nu^{-i}\lambda^{-(\ell-i)}(-\overline{\lambda})^{-j}(x\overline{\nu}^{-1})^{\ell-j}\\
        =&\frac{(\ell-v)!(\ell+v)!|\nu|^{-2\ell}\nu^{v}\lambda^{-v}}{(\ell!)^{2}(|\lambda|^{-2}+x^{2}|\nu|^{-2})^{2\ell+1}}\sum_{j=\max(0,-v)}^{\min(\ell,\ell-v)}(-1)^{j}\binom{\ell}{j}\binom{\ell}{j+v}|\nu|^{2j}|\lambda|^{-2j}x^{2\ell-v-2j}\\
    \end{align*}
    Using the fact that 
    \[\int_{x\in\R}\frac{x^{2m}}{(|\lambda|^{-2}+x^{2}|\nu|^{-2})^{2\ell+1}}=\pi|\lambda|^{4\ell-2m+1}|\nu|^{2m+1}2^{-4\ell}\frac{(4\ell-2m)!(2m)!}{(2\ell-m)!m!(2\ell)!},\]
    we can integrate $f_{\ell,\infty}(\gamma_{0}n(-xc_{2},0)g)$ over $x\in\R$,
    and its $[u_{1}^{\ell-v}][u_{2}^{\ell+v}]$ coefficient is $0$ when $v$ is odd,
    otherwise,
    it is 
    %\[\frac{(\ell-v)!(\ell+v)!|\nu|^{-2\ell}\nu^{v}\lambda^{-v}}{(\ell!)^{2}}\sum_{j=\max(0,-v)}^{\min(\ell,\ell-v)}(-1)^{j}\binom{\ell}{j}\binom{\ell}{j+v}|\nu/\lambda|^{2j}\pi |\lambda|^{2\ell+v+2j+1}|\nu|^{2\ell-v-2j+1}2^{-4\ell}\frac{(2\ell+v+2j)!(2\ell-v-2j)!}{(\ell+v/2+j)!(\ell-v/2-j)!(2\ell)!}\]
    \begin{align*}
        \frac{\pi|\lambda|^{2\ell+1}|\nu|}{(2\ell)!(\ell!)^{2}2^{4\ell}}\cdot (\ell-v)!(\ell+v)!(\nu/\lambda)^{v}|\nu/\lambda|^{-v}\sum_{j=\max(0,-v)}^{\min(\ell,\ell-v)}(-1)^{j}\binom{\ell}{j}\binom{\ell}{j+v}\frac{(2\ell+v+2j)!(2\ell-v-2j)!}{(\ell+v/2+j)!(\ell-v/2-j)!}.    
    \end{align*}
    The inner sum is symmetric under $v\mapsto -v$,
    and when $v$ is non-negative,
    then 
    it equals
    \begin{align*}
        &\sum_{j=0}^{\ell-v}(-1)^{j}\binom{\ell}{j}\binom{\ell}{j+v}\frac{(2\ell+v+2j)!(2\ell-v-2j)!}{(\ell+v/2+j)!(\ell-v/2-j)!}\\
        =&\frac{2^{4\ell}}{\pi}\Gamma(\ell+v/2+1/2)\Gamma(\ell-v/2+1/2){_{3}F_{2}(-\ell,-\ell+v,\ell+v/2+1;v+1,-\ell+v/2+1/2;1)}.
    \end{align*}
    So we have 
    {\scriptsize\begin{align*}
        f_{0,\infty}(s)=\frac{|\lambda(g)|^{2\ell+1}|\nu(g)|}{(2\ell)!}\sum_{\substack{-\ell\leq v\leq \ell\\ v\text{ is even}}}\binom{\ell+|v|}{\ell}\Gamma(\ell+v/2+1)\Gamma(\ell-v/2+1)\left(\frac{\nu(g)\overline{\lambda(g)}}{\overline{\nu(g)}\lambda(g)}\right)^{v/2}{_{3}F_{2}(-\ell,-\ell+v,\ell+v/2+1;v+1,-\ell+v/2+1/2;1)}.
    \end{align*}}
   \end{proof}
   \textcolor{red}{There must be some mistake in this computation. I need to check it.}
   
\subsection{Rank 1 Fourier coefficients}
\label{section rank 1 Fourier coefficients}
\subsection{Rank 2 Fourier coefficients}
\label{section rank 2 Fourier coefficients}
One decomposes $E_{2}^{T}(1,s)$ as $\prod_{v}E_{2,v}^{T}(s)$.
\subsubsection{Finite places}
\label{section finite place rank 2 Fourier} 
\begin{align*}
    E_{2,p}^{T}(s)&=\int_{\bff{N}(\Q_{p})}\chi_{T}^{-1}(n)f_{p}(w_{2}n,s)dn\\
    &=\int_{\bff{N}(\Q_{p})}\int_{t\in E_{p}^{\times}}\chi_{T}^{-1}(n)|t|^{s}\Phi_{p}(t\cdot b_{1}w_{2}n)dtdn\\
    &=\int_{v\in \bff{V}_{0}(\Q_{p})}\int_{\substack{x\in E_{p}\\ \overline{x}=-x}}\int_{t\in E_{p}^{\times}}\chi_{T}^{-1}(v)|t|^{s}\Phi_{p}\left(t\left(\left(-\lrangle{v}{v}/2+x\right)b_{1}+b_{2}+v\right)\right)dtdxdv
\end{align*}
Take $\oo_{E_{p}}$ to be $\Z_{p}\otimes_{\Z}\oo_{E}\subseteq E_{p}$,
$\frakk{p}$ to be its maximal ideal,
and $\call{V}$ an $\oo_{E_{p}}$-lattice of $V_{0}(\Q_{p})=V_{0}\otimes_{\Q}\Q_{p}$ 
such that $\call{V}\otimes_{\oo_{E_{p}}}\left(\oo_{E,p}/\frakk{p}\right)$ is a non-degenerate Hermitian space over $\oo_{E_{p}}/\frakk{p}$.
Assume that $\Phi_{p}$ is the characteristic function of the lattice 
$\oo_{E_{p}}b_{1}\oplus \call{V}\oplus \oo_{E_{p}}b_{2}$.

\textcolor{red}{Write down a precise definition of $\call{V}$?}
\begin{lemma}\label{lemma finite rank 2 coefficients}
    \begin{enumerate}
        \item If $p$ splits in $E$,
        then 
        \[E_{2,p}^{T}(s)=\sum_{r_{1},r_{2}\geq 0}p^{-(r_{1}+r_{2})s+\min(r_{1},r_{2})}\left(\int_{v\in (p^{-r_{1}},p^{-r_{2}})\call{V}}\chi_{T}^{-1}(v)\rmm{Char}(p^{\rmm{max}(r_{1},r_{2})}\lrangle{v}{v}\in \Z_{p})dv\right).\]
        \item If $p$ is inert in $E$,
        then
        \[E_{2,p}^{T}(s)=\sum_{r\geq 0}p^{-2rs+r}\left(\int_{v\in p^{-r}\call{V}}\chi_{T}^{-1}(v)\rmm{Char}\left(p^{r}\lrangle{v}{v}/2\in \Z_{p}\right)dv\right).\] 
        \item If $p$ is ramified in $E$, then 
    \end{enumerate}
\end{lemma}
\begin{proof}
    (1) If $p$ splits in $E$, then
$E_{p}=\Q_{p}\times\Q_{p}$:
\begin{align*}
    E_{2,p}^{T}(s)=&\sum_{r_{1}\geq 0,\,r_{2}\geq 0}|p|^{(r_{1}+r_{2})s}\int_{v\in (p^{-r_{1}},p^{-r_{2}})\call{V}}\chi_{T}^{-1}(v)\left(\int_{\substack{x\in (p^{-r_{1}},p^{-r_{2}})\oo+\lrangle{v}{v}/2\\ x+\overline{x}=0}}dx\right)dv
\end{align*}
If we write $x=(y,-y)\in E_{p}$,
then $x\in (p^{-r_{1}},p^{-r_{2}})\oo+\lrangle{v}{v}/2$ is equivalent to 
\[y\in p^{-r_{1}}\Z_{p}+\lrangle{v}{v}/2,\,-y\in p^{-r_{2}}\Z_{p}+\lrangle{v}{v}/2.\]
There are such $x$ only if $\lrangle{v}{v}\in p^{-\max(r_{1},r_{2})}\Z_{p}$,
If $r_{1}\geq r_{2}$,
then $y\in p^{-r_{2}}\widetilde{y}-\lrangle{v}{v}/2$ for some $\widetilde{y}\in \Z_{p}$.
Since the measure on the line $\{x=(y,-y)\in E_{p}\}$ is the normalized Haar measure $dy$ on $\Q_{p}$,
one has $dy=p^{r_{2}}d\widetilde{y}$,
and 
\[\int_{x=(y,-y)\in p^{-r_{1}}\Z_{p}\times p^{-r_{2}}\Z_{p}+\lrangle{v}{v}/2}dx=\int_{\widetilde{y}\in \Z_{p}}p^{r_{2}}d\widetilde{y}=p^{r_{2}}=p^{\min(r_{1},r_{2})}.\]
Hence
\[E_{2,p}^{T}(s)=\sum_{r_{1},r_{2}\geq 0}p^{-(r_{1}+r_{2})s+\min(r_{1},r_{2})}\left(\int_{v\in (p^{-r_{1}},p^{-r_{2}})\call{V}}\chi_{T}^{-1}(v)\rmm{Char}(p^{\rmm{max}(r_{1},r_{2})}\lrangle{v}{v}\in \Z_{p})dv\right).\]

(2) If $p$ is inert in $E$,
then $E_{p}$ is an unramified quadratic field extension of $\Q_{p}$, 
and 
\begin{align*}
    E_{2,p}^{T}(s)=\sum_{r\geq 0}|p|_{E_{p}}^{rs}\int_{v\in p^{-r}\call{V}}\chi_{T}^{-1}(v)\rmm{Char}(p^{r}\lrangle{v}{v}\in\Z_{p})\left(\int_{x\in\lrangle{v}{v}/2+p^{-r}\oo_{E_{p}}}\rmm{Char}(x+\overline{x}=0)dx\right)dv
\end{align*}
Choose a unit $u\in \oo_{E_{p}}^{\times}$ with $u+\overline{u}=0$,
then the line $\set{x\in E_{p}}{x+\overline{x}=0}$ can be written as $\{yu,\,y\in \Q_{p}\}$,
with the normalized Haar measure $dy$.
There exist elements $yu\in \lrangle{v}{v}/2+p^{-r}\oo_{E_{p}}$ if and only if $\lrangle{v}{v}/2\in p^{-r}\Z_{p}$.
When we have $\lrangle{v}{v}/2\in p^{-r}\Z_{p}$,
any element $yu\in\lrangle{v}{v}/2+p^{-r}\oo_{E_{p}}$ is of the form 
\[yu=\lrangle{v}{v}/2+p^{-r}\left(-p^{r}\lrangle{v}{v}/2+\widetilde{y}u\right),\,\widetilde{y}\in \Z_{p},\]
thus one has 
\begin{align*}
    \int_{x\in\lrangle{v}{v}/2+p^{-r}\oo_{E_{p}}}\rmm{Char}(x+\overline{x}=0)dx&=\rmm{Char}(p^{r}\lrangle{v}{v}/2\in\Z_{p})|p|_{\Q_{p}}^{-r}\int_{\Z_{p}}d\widetilde{y}\\
    &=p^{r}\rmm{Char}(p^{r}\lrangle{v}{v}/2\in\Z_{p}),    
\end{align*}
which gives us the desired identity.
\end{proof}     


\textbf{If $p$ is ramified in $E$:}
(Assume $p\neq 2$)
Then $E_{p}$ is a ramified quadratic field extension of $\Q_{p}$,
and we choose an uniformizer $\varpi$ of $\frakk{p}\subset\oo_{E_{p}}$ such that $\varpi+\overline{\varpi}=0$and $\varpi^{2}\in p\Z_{p}^{\times}$.
The integral for $E_{2,p}^{T}$ can be rewritten as:
\begin{align*}
    E_{2,p}^{T}(s)=&\sum_{r\geq 0}|\varpi|^{rs}\int_{v\in \varpi^{-r}\call{V}}\chi_{T}^{-1}(v)\rmm{Char}(p^{\lfloor r/2\rfloor}\lrangle{v}{v}\in \Z_{p})\left(\int_{x\in\Q_{p}}\rmm{Char}(x\varpi-\frac{\lrangle{v}{v}}{2}\in \varpi^{-r}\oo_{E_{p}})dx\right)dv\\
    =&\sum_{r\geq 0}|\varpi|^{rs}p^{\lceil r/2 \rceil}\left(\int_{v\in \varpi^{-r}\call{V}}\chi_{T}^{-1}(v)\rmm{Char}(p^{\lfloor r/2\rfloor}\lrangle{v}{v}\in\Z_{p})dv\right)
\end{align*} 
\subsubsection{Archimedean place}
\label{section archimedean place rank 2 Fourier}
\begin{rmk}\label{rmk complex absolute value}
    In this article,
    when we write $|z|$ for $z\in\C$,
    it means the \emph{norm of $z$ with respect to the extension $\C/\R$},
    \emph{i.e.\,}$|z|=z\overline{z}$ instead of the usual modulus $\sqrt{z\overline{z}}$.
    I admit that this is somehow strange and confusing,
    but I will fix this problem if this draft could become a paper...
\end{rmk}
We first analyze the function $f_{\ell,\infty}(w_{2}n,s=\ell+1)$:
\begin{lemma}
    \label{lemma rank 2 values of archimedean section}
    For any $v\in V_{0}\otimes_{\Q}\R$ and $x\in \R$,
    we set:
    \[\alpha(v,x)=-\frac{\lrangle{v}{v}}{2}+ix+1,\,\beta(v)=\sqrt{2}\lrangle{v}{u_{2}},\]
    then we have 
    \[f_{\ell,\infty}(w_{2}n(v,ix),s)=\frac{(\alpha u_{1}+\beta u_{2})^{\ell}(-\overline{\beta}u_{1}+\overline{\alpha}u_{2})^{\ell}}{(|\alpha(v,x)|+|\beta(v)|)^{\ell+s}(\ell!)^{2}}.\]
\end{lemma}
\begin{proof}
    One has 
    \[b_{1}w_{2}n(v,ix)=(-\lrangle{v}{v}/2+ix)b_{1}+b_{2}+v.\]
    Suppose that we can decompose $w_{2}n(v,ix)$ as $pk$ for some $p\in \bff{P}(\R)$ and $k\in K_{\infty}$,
    then 
    \begin{equation}\label{eqn searching Iwasawa decomposition}
         b_{1}w_{2}n(v,ix)k^{-1}=b_{1}p=\nu(p)^{-1}b_{1}.
    \end{equation}  
    Let $k_{+}$ be the factor of $k$ in $\rmm{U}(V_{2}^{+})$,
    and $v=v_{+}+v_{-}\in \C u_{2}\oplus \rmm{Span}_{\C}(v_{1},\ldots,v_{n-1})$.
    Taking the $V_{2}^{+}$ components of \Cref{eqn searching Iwasawa decomposition},
    one gets:
    \[\frac{1}{\sqrt{2}}\left(\alpha(v,x)u_{1}+\beta(v)u_{2}\right)=\left[\frac{1}{\sqrt{2}}\left(-\frac{\lrangle{v}{v}}{2}+ix+1\right)u_{1}+v_{+}\right]k_{+}^{-1}=\nu(p)^{-1}\frac{u_{1}}{\sqrt{2}}.\]
    The norms of both sides give us the identity $|\alpha(v,x)|+|\beta(v)|=|\nu(p)|^{-1}$.
    One may assume that 
    $\nu(p)^{-1}=\sqrt{|\alpha(v,x)|+|\beta(v)|}$,
    then in the basis of $u_{1},u_{2}$,
    the element $k^{+}\in \rmm{U}(V_{2}^{+})$ can be written as the Hermitian matrix 
    \[\frac{1}{\sqrt{|\alpha(v,x)|+|\beta(v)|}}\left(\begin{matrix}
        \alpha(v,x) & -\overline{\beta(v)}\\
        \beta(v) & \overline{\alpha(v,x)}
    \end{matrix}\right).\]
    Plug $k^{+}$ and $\nu(p)$ into $f_{\ell,\infty}(w_{2}n(v,ix),s)=f_{\ell,\infty}(pk,s)=|\nu(p)|^{s}[u_{1}^{\ell}][u_{2}^{\ell}].k^{+}$,
    and we get the desired value.
\end{proof}
Let $I_{0}(T;\ell)$ be the coefficient of $[u_{1}^{\ell}][u_{2}^{\ell}]$ in $E_{2,\infty}^{T}(s=\ell+1)$,
which can be written as 
\begin{align*}
    I_{0}(T;\ell)=&\int_{v\in V_{0}\otimes_{\Q}\R}\int_{x\in \R}\chi_{T}^{-1}(v)\sum_{k=0}^{\ell}\frac{{\ell \choose k}{\ell\choose k}(x^{2}+A)^{k}(-B)^{\ell-k}}{(x^{2}+A+B)^{2\ell+1}}dxdv\\
    =&\int_{v\in V_{0}\otimes \R}\chi_{T}^{-1}(v)\sum_{k=0}^{\ell}(-B)^{\ell-k}{\ell\choose k}^{2}\int_{x\in \R}\frac{(x^{2}+A)^{k}}{(x^{2}+A+B)^{2\ell+1}}dxdv
\end{align*}
where $A=|\alpha|-x^{2}=\left(1-\lrangle{v}{v}/2\right)^{2}$, and $B=|\beta|=2|\lrangle{v}{u_{2}}|$.
\begin{lemma}
    \label{lemma integral over center of Heisenberg}
    For any real number $C,D$
    and two natural numbers $m<n$,
    we have 
    \[\int_{\R}\frac{(x^{2}+C)^{m}}{(x^{2}+D)^{n}}dx=\frac{D^{m-n+1/2}}{(n-1)!}\sum_{k=0}^{m}{m\choose k}\left(\frac{C}{D}\right)^{m-k}\Gamma(k+1/2)\Gamma(n-k-1/2).\]
\end{lemma}
\begin{proof}
    An exercise of calculus.
\end{proof}
Now \Cref{lemma integral over center of Heisenberg} tells us $I_{0}(T;\ell)$ is the Fourier transform of the function 
\begin{align*}
    F_{0,\ell}(v):=\sum_{k=0}^{\ell}(-B)^{\ell-k}{\ell\choose k}^{2}\frac{(A+B)^{k-2\ell-1/2}}{(2\ell)!}\sum_{j=0}^{k}{k\choose j}\left(\frac{A}{A+B}\right)^{k-j}\Gamma(j+1/2)\Gamma(2\ell+1/2-j)
\end{align*}
Set $z=B/(A+B)$,
then this function becomes 
\[\frac{2^{-4\ell}}{(2\ell)!}\pi(A+B)^{-\ell-1/2}\sum_{k=0}^{\ell}\sum_{j=0}^{k}{\ell\choose k}^{2}{k\choose j}\frac{(2j)!(4\ell-2j)!}{j!(2\ell-j)!}(-z)^{\ell-k}(1-z)^{k-j}\]
We write:
\[\sum_{k=0}^{\ell}\sum_{j=0}^{k}c_{j,k}(-z)^{\ell-k}(1-z)^{k-j}=\sum_{r=0}^{\ell}(-1)^{r}C(r)z^{r},\]
where $c_{j,k}={\ell\choose k}^{2}{k\choose j}\frac{(2j)!(4\ell-2j)!}{j!(2\ell-j)!}$.
The term $(-z)^{\ell-k}(1-z)^{k-j}$ has a non-zero $z^{r}$ term if and only if 
$\ell-k\leq r\leq \ell-j$,
thus 
\[(-1)^{r}C(r)=\sum_{j=0}^{\ell-r}\sum_{k=\ell-r}^{\ell}c_{j,k}{{k-j}\choose {r-\ell+k}}(-1)^{\ell-k+r-\ell+k}=(-1)^{r}\sum_{j=0}^{\ell-r}\sum_{k=\ell-r}^{\ell}c_{j,k}{{k-j}\choose {r-\ell+k}}.\]
To compute $C(r)$,
we need the following lemma:
\begin{lemma}
    \label{lemma combinatorics summations}
    \begin{enumerate}
        \item For integers $0\leq a\leq b$, one has \[\sum_{i=a}^{b}{b\choose i}{{b-a}\choose{b-i}}={{2b-a}\choose b}.\]
        \item For any integer $0\leq r\leq \ell$, one has 
        \[\sum_{i=0}^{\ell-r}\frac{\binom{2\ell}{i}\binom{\ell-r}{i}}{\binom{4\ell}{2i}}=2^{2\ell-2r}\frac{\binom{2\ell+2r}{\ell+r}}{\binom{4\ell}{2\ell}}.\]
    \end{enumerate}
\end{lemma}
\begin{proof}
    The identity in (1) is obvious.
    For the identity in (2),
    the LHS can be rewritten as:
    \begin{align*}
        \sum_{i=0}^{\ell-r}\frac{(-\ell+r)_{j}(1/2)_{j}}{(-2\ell+1/2)_{j}j!}={_{2}F_{1}}(-(\ell-r),1/2;-2\ell+1/2;1),
    \end{align*}
    where $(x)_{j}$ is the (rising) Pochhammer symbol,
    and $_{2}F_{1}$ is the hypergeometric function.
    By Chu-Vandermond identity,
    this value of hypergeometric function is
    \[\frac{(-2\ell)_{\ell-r}}{(-2\ell+1/2)_{\ell-r}}=2^{\ell-r}\frac{\frac{(2\ell)!}{(\ell+r)!}}{\frac{(4\ell-1)!!}{(2\ell+2r-1)!!}}=2^{2\ell-2r}\frac{\binom{2\ell+2r}{\ell+r}}{\binom{4\ell}{2\ell}}.\qedhere\]
\end{proof}
Now we return to the value of $C(r)$:
\begin{align*}
    C(r)&=\sum_{j=0}^{\ell-r}\sum_{k=\ell-r}^{\ell}\frac{(\ell!)^{2}(2j)!(4\ell-2j)!}{k!((\ell-k)!)^{2}(j!)^{2}(r-\ell+k)!(\ell-r-j)!(2\ell-j)!}\\
    &=\sum_{j=0}^{\ell-r}\frac{\ell!(2j)!(4\ell-2j)!}{r!(j!)^{2}(\ell-r-j)!(2\ell-j)!}\sum_{k=\ell-r}^{\ell}\binom{\ell}{k}\binom{r}{\ell-k}\\
    (\text{by (1) of \Cref{lemma combinatorics summations}})&=\frac{\ell!}{r!}\cdot\frac{(\ell+r)!}{\ell!r!}\sum_{j=0}^{\ell-r}\frac{(2j)!(4\ell-2j)!}{(j!)^{2}(\ell-r-j)!(2\ell-j)!}\\
    &=\frac{(\ell+r)!}{(r!)^{2}}\cdot \frac{(4\ell)!}{(\ell-r)!(2\ell)!}\sum_{j=0}^{\ell-r}\frac{\binom{2\ell}{j}\binom{\ell-r}{j}}{\binom{4\ell}{2j}}\\
    (\text{by (2) of \Cref{lemma combinatorics summations}})&=\frac{(\ell+r)!}{(r!)^{2}}\cdot\frac{(4\ell)!}{(\ell-r)!(2\ell)!}\cdot 2^{2\ell-2r}\frac{\frac{(2\ell+2r)!}{((\ell+r)!)^{2}}}{\frac{(4\ell)!}{((2\ell)!)^{2}}}\\
    &=2^{2\ell-2r}\frac{(2\ell)!(2\ell+2r)!}{(r!)^{2}(\ell+r)!(\ell-r)!}.
\end{align*}
Putting the value of $C(r)$ into $F_{0,\ell}$,
we have 
\begin{align*}
    F_{0,\ell}(v)=&\frac{2^{-4\ell}\pi}{(2\ell)!(A+B)^{\ell+1/2}}\sum_{r=0}^{\ell}(-1)^{r}2^{2\ell-2r}\frac{(2\ell)!(2\ell+2r)!}{(r!)^{2}(\ell+r)!(\ell-r)!}\\
    =&\frac{2^{-3\ell}(2\ell)!\pi}{(\ell!)^{2}(A+B)^{\ell+1/2}}\sum_{r=0}^{\ell}\frac{(-\ell)_{r}(\ell+1/2)_{r}}{1_{r}}\frac{z^{r}}{r!}\\
    =&\frac{2^{-3\ell}(2\ell)!\pi}{(\ell!)^{2}(A+B)^{\ell+1/2}}\cdot {_{2}F_{1}}(-\ell,\ell+1/2;1;z).
\end{align*}
We have shown the following result:
\begin{prop}\label{prop archimedean rank 2 Fourier}
    For $T\in V_{0}$ with $\lrangle{T}{T}>0$,T
    the coefficient $I_{0}(T;\ell)$ of $[u_{1}^{\ell}][u_{2}^{\ell}]$ in $E_{2,\infty}^{T}(s=\ell+1)$ is the Fourier transform of the function:
    \[F_{0,\ell}(v)=\frac{2^{-3\ell}(2\ell)!\pi}{(\ell!)^{2}(A+B)^{\ell+1/2}}\cdot {_{2}F_{1}}(-\ell,\ell+1/2;1;\frac{B}{A+B}),\]
    where $A=\left(1-\frac{\lrangle{v}{v}}{2}\right)^{2}$ and $B=2|\lrangle{v}{u_{2}}|^{2}$.
\end{prop}

Let's recall the following result for quadratic spaces by Pollack:
\textcolor{red}{I know there are notation problems again... just let it be like this for now}
\begin{prop}\label{prop Fourier transform of K Bessel functions}
    ($v=0$ case of \cite[Proposition 4.5.3]{Pollack_Modular_forms_orthogonal_rank3})
    Let $(V^{\prime},(\,,\,))=V_{2}\oplus V_{n}$ be a non-degenerate quadratic space over $\R$ of signature $(2,n)$,
    and $v_{1},v_{2}$ an orthonormal basis of $V_{2}$.
    Set 
    \[I_{0}(x;\ell)=\int_{V^{\prime}}e^{i(\omega,x)}\rmm{Char}(q(\omega)>0)q(\omega)^{\ell-n/2}K_{0}(\sqrt{2}|(\omega,v_{1}+\sqrt{-1}v_{2})|)d\omega.\]
    This integral is absolutely convergent and we have 
    \begin{align*}
        I_{0}(x;\ell)=(2\pi)^{(n+2)/2}2^{\ell-1-n/2}\Gamma(\ell+1)\Gamma(\ell+1-n/2)F_{4}(\ell+1,\ell+1;\ell+1;1;-\|x_{n}\|^{2}/2;-\|x_{2}\|^{2}/2),
    \end{align*}
    where $x=x_{2}+x_{n}$,
    and $F_{4}(a,b;c;d;x;y)$ is Appell's hypergeometric function.
\end{prop} 
Now we look at how this result fits into our unitary group setting.
One take $V^{\prime}$ to be $(V_{0},\rmm{Re}(\lrangle{\,}{\,}))$,
as a quadratic space of signature $(2,2n-2)$,
$v^{\prime}_{1}=u_{2}$,
$v^{\prime}_{2}=i u_{2}$,
then 
\begin{align*}
    I_{0}(x;\ell)&=\int_{T\in V_{0},\,\lrangle{T}{T}>0}e^{i\rmm{Re}\lrangle{T}{x}}\lrangle{T}{T}^{\ell-n+1}K_{0}(\sqrt{2}|\lrangle{T}{u_{2}}|)dT\\
    &=\int_{T\in V_{0},\,\lrangle{T}{T}>0}e^{2\pi i \rmm{Re}\lrangle{T}{x}}(2\pi)^{2\ell-2n+2}\lrangle{T}{T}^{\ell-n+1}K_{0}(2\sqrt{2}\pi |\lrangle{T}{u_{2}}|)(2\pi)^{2n}dT\\
    &=(2\pi)^{2\ell+2}\int_{T\in V_{0},\,\lrangle{T}{T}>0}e^{2\pi i \rmm{Re}\lrangle{T}{x}}\lrangle{T}{T}^{\ell-n+1}K_{0}(2\sqrt{2}\pi |\lrangle{T}{u_{2}}|)dT,
\end{align*}
here $|z|=\sqrt{z\bar{z}}$,
and for any $\ell>n-1$ we have
\begin{align*}
    I_{0}(x;\ell)=(2\pi)^{n}2^{\ell-n}\ell!(\ell-n+1)!(A+B)^{-(\ell+1)/2}{_{2}F_{1}}(-\ell/2,(\ell+1)/2;1;B/(A+B)).
\end{align*}
Replace $\ell$ by $2\ell$ (now $\ell>\frac{n-1}{2}$),
and we have 
\[I_{0}(x;2\ell)=(2\pi)^{n}2^{2\ell-n}(2\ell)!(2\ell-n+1)!(A+B)^{-\ell-1/2}\cdot{_{2}F_{1}}(-\ell,\ell+1/2;1;B/(A+B)).\]
Comparing this with \Cref{prop archimedean rank 2 Fourier},
we get the following theorem:
\begin{thm}\label{thm rank 2 Fourier archimedean generalised Whittaker}
    For $T\in V_{0}$ with $\lrangle{T}{T}>0$ and $\ell>\frac{n-1}{2}$,
    we have
    \[I_{0}(T;\ell)=\frac{2^{-\ell-2}\pi^{4\ell-n+3}\lrangle{T}{T}^{2\ell-n+1}}{\ell!(2\ell-n+1)!}\call{W}_{T}(1).\]
\end{thm}

\printbibliography
\end{document}